\section*{Objectives}

The aim of this homework is to read and briefly analyze following papers and articles: \cite{LR1}, \cite{LR2}, \cite{LR3}, \cite{SLR1} and \cite{SLR2}. There are 2 types of the works - those which are dealing with Literature Review (LR) and those which are dealing with Systematic Literature Review (SLR). In the first case it is needed just to evaluate quality of them and write a brief note. The main focus of the second category is to determine and identify the review protocol, the search strategy and to discuss possible differences and commonalities. Both will be described in section \ref{Report}.

Also, it is required to comment the main differences of LRs and SLRs based on observations from these works, which will be done in section \ref{Conclusions}.

\section*{Review report}\label{Report}
\subsection*{Literature Review}
\begin{itemize}
	\item \textbf{A Literature Review on Cyber Security \cite{LR2}} \\
	The work is aiming on data protection in terms of Cyber Security with the focus on one time password, event log analysis, malicious attack detection and virtualization. The work has many grammar errors and it is not very cited by other works on the Internet even it is from year 2007. It does not explain properly the main concepts which is aiming at. They also did not explain or not even cited many minor mechanisms they are dealing with. The authors didn't compared relations between sources, gaps or disagreements and in the terms of Cuber Security there certainly are. Not mention that there are far more techniques which are dealing with data protection. Regarding their conclusion they mention that in this LR they summarized protection of privacy, but it is done very poorly. The LR work should have better structure and should be written in more logical way as well as with the grammar without errors. But at least they classify few of very important areas and they collected relevant works which are also classified by this topics. 
	 
	\item \textbf{Literature Survey on Automated Person Identification Techniques \cite{LR3}} \\
	This work is aiming at automatic person identification. For the first insight it seems to be well structured with the correct grammar and even person which is not entirely familiar with those principles can understand main ideas. But not all of them in some more detail, which is one of my critic about this work - it is dealing just with possibilities and concepts from a very far distance. The work is from year 2013 and it is not cited by other works on the Internet so the LR is not very popular. This works provide some basic and very brief taxonomy of Biometric systems which was their goal. They also compare all of techniques in tabular format. They also provided some very brief conclusion about the findings. They could provide more resources and this work definitely suffers from that a lot. They described techniques but didn't provide the main and most popular works about each individual subarea. They did not explain more about techniques in the order to be more familiar with different approaches and they did not deal with the relations between sources, critical gaps and disagreements. They basically just explained the main principles (in same cases quite poorly) and provide some information sources in a well structured and easy to understand format. I would never choose to read this (nor previous) work because the information value as well as final quality is very low.
	
	\item \textbf{A Review on Deep Learning Techniques Applied to Semantic Segmentation\cite{LR1}} \\
	The last work in this category deals with Deep Learning Techniques in one specific area. It is considered to be one of the first works about this particular topic and according to Internet it is cited more even to the fact that it is relatively young, from year 2017. The authors provided an exhausting list of information sources as well as their observations in final section with conclusion. It also deals with datasets used in previous works and they are comparing them. They provide a lot of explanations as well as in the text itself and also in other resources, not just in other papers. They also explained the current situation and why their LR is important as well as some of their observations and contributions. It goes in more depth with details, not just brief principles so the reader can understand it better and in more detail. They also proposed some future research directions and made new observations from known (and many times common) datasets and evaluation of different methods. I consider this work to be written with quality, well format and if I was starting to studying this area, this would be probably one of the first LR I would read even to the fact that it has some of the properties of SLR.
\end{itemize}

\subsection*{Systematic Literature Review}
\begin{itemize}
\item \textbf{A SLR of empirical evidence on computer games and serious games \cite{SLR2}}
	\begin{enumerate}
	  	\item \textit{The review protocol} \\
	  	In this work the authors are focusing on gathering evidence about the positive impacts and outcomes of computer games with respect to learning since there are many studies about this topic, but many of them are just speculations and there are not too many arguments which would support outcomes of these works. So this work has a clear goal.	So they have a clear research question and also they specify a method how to achieve their goal. They are describing data collection and its analysis. They performed a ranking of quality of articles they were analyzing and they compared resulting articles according to different criteria. After that they are analyzing the chosen articles from more aspects. They compared high quality works and they discussed similarities and differences too. Furthermore they described what to include to their work and what to exclude by specifying various criteria.
		\item \textit{The search strategy} \\
		The authors gathered information resources through online databases and they enumerated them as it was explained in previous section. They considered a specific time period of when the studies and experiments were performed - to be more specific they chose only studies in between January 2004 and February 2009. The authors also provided an exact search terms in their search strategy, to be more precise it consists of two patterns: 
		\begin{itemize}
			\item ("computer games" OR "video games" OR "serious games" OR "simulation games" OR "games-based learning" OR "MMOG" OR "MMORPG" OR "MUD" OR "online games")
			\item AND ("evaluation" OR "impacts" OR "outcomes" OR "effects" OR "learning" OR "education" OR "skills" OR "behaviour" OR "attitude" OR "engagement" OR "motivation" OR "affect")
		\end{itemize}
	\end{enumerate}
	
\item \textbf{Automatic Assessment of Depression Based on Visual Cues - A Systematic Review \cite{SLR1}}
	\begin{enumerate}
	  	\item \textit{The review protocol} \\
	  	The research question has been set and the authors made observations and wrote conclusions during a given SLR. They found a trend towards utilizing high level features and they provided a strong arguments and proofs for supporting that. The main questions were addressed to following: a) if the video-based depression assessment can assist the diagnosis and monitoring of the disorder and b) if visual cues alone are sufficient or if they need to be supplemented by information from other modalities. This work includes summary and comparison of approaches in previous studies they were considered. They gathered sources from databases they specified and they also used datasets which were described in a detailed way. 
		\item \textit{The search strategy} \\
		The period of interest for this review was taken from year 2007 until April of 2017, but there were also considered some publications from 2005, which are also mentioned in the text and references. In this work the authors gathered resources according to following 2 set of keywords, each one for different usage:
		\begin{itemize}
			\item "Depression", "Facial Expression", "Non-verbal communication", "Image Processing", "machine learning", "Biomedical Imaging", "Face", "Emotion", "Computer Vision"
			\item "Depression", "Definition", "Types", "Frequency or rate", "Diagnostic tests", "Etiology and risk factors", "Predictability"
		\end{itemize}
	\end{enumerate}
	\item \textit{Differences and commonalities} \\
		Some of the \textbf{differences} are that in the first case the authors clearly and explicitly defined the pattern for the search strategy. In second case the authors just mentioned all the keywords, but they did not defined relationship between them. Also in the first case the authors worked only with a specific time period which ended a few years before they wrote the paper, but in the second case it can be noticed that they considered data from some period (also with consideration to some older studies) until the actual date back then so they covered also newer works.
		Regarding \textbf{commonalities}, it can be seen that they both worked with a big variety of databases and datasets and they set the rules for they inclusion and exclusion. They also set some research questions and they focused on the answers together with review with a big quality. They provided a great comparison in tabular way regarding older works and approaches.
\end{itemize}

\section*{Conclusions}\label{Conclusions}
The main observations regarding differences between LR and SLR approaches are, that LR can contain information and explanations just on the surface, the authors in such works in some cases don't go in too much detail. There can be no (research) questions at all and it can be just a short summary of some works about a particular topic. According to my observations, in LR the authors are not always comparing all the approaches and related studies, which is a total opposite to SLR. 

On the other hand, SLR has an exact plan how to perform a review: choosing relavant keywords, rules for inclusion and exclusion of obtained works, but also scientific databases and if needed also datasets they will use in their systematic review. They have a bigger purpose with their work, not just fill in the reader to the problematic. They usually go in more detail and compare previous approaches, studies and also have their own conclusion from the observations. They can answer to a given questions or they can also propose to look on some things from a different point of view, which means that it is not just sophisticated summary of previous work, but there are also some new acknowledgement.